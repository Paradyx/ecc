% Dieser Text ist urheberrechtlich gesch�tzt
% Er stellt einen Auszug eines von mir erstellten Referates dar
% und darf nicht gewerblich genutzt werden
% die private bzw. Studiums bezogen Nutzung ist frei
% Nov. 2007
% Autor: Sascha Frank 
% Universit�t Freiburg 
% www.informatik.uni-freiburg.de/~frank/
\documentclass{beamer}
\setbeamertemplate{navigation symbols}{}
\usetheme{Bergen}
%\usepackage{ngerman}
%\beamersetuncovermixins{\opaqueness<1>{25}}{\opaqueness<2->{15}}

\usepackage[utf8]{inputenc}

\usepackage{makecell}
\renewcommand\theadalign{cb}
\renewcommand\theadfont{\bfseries}
\renewcommand\theadgape{\Gape[4pt]}
\renewcommand\cellgape{\Gape[4pt]}

\usepackage{colortbl}

%\usepackage{cite}
\usepackage{caption}

\usepackage[style=numeric]{biblatex}
\addbibresource{literature.bib}

\usefonttheme[onlymath]{serif}

\usepackage{graphics}

\begin{document}
\title{Implementation of an ECC with M-511} 
\subtitle{CS488 - Introduction to IT Security}
\author{Yoann Kehler, Duong Ta}
\date{\today} 

\begin{frame}
\titlepage
\end{frame} 

\begin{frame}{Outline}
\begin{itemize}
	\item Why is non-elliptic curve cryptography not enough?
	\item What is an elliptic curve?
	\item Why M-511?
\end{itemize}
\end{frame} 
\begin{frame}{The Discrete Logarithm Problem\cite{werner2013elliptische}}
	Finite abelian Group, with multiplication\\
	\[(P,Q) \mapsto P \cdot Q\]
	\pause
	An element $P$ generates a cyclic Subgroup with order $n$\\
	\[\langle P \rangle = \{P^k : k \in \mathbb{Z} \} \]
	\[n = ord(P) = |\langle P \rangle| \]\\ \vspace{0.5cm}
	\pause
	\textbf{The Discrete Logarithm Problem (DLP)}: \\
	For a given $P$ and $Q \in \langle P \rangle$ determine $k$ s.t. \\
	\[Q=P^k\]
\end{frame}
\begin{frame}{The Discrete Logarithm Problem\cite{werner2013elliptische}}
Finite abelian Group, with \textcolor[rgb]{1.00,0.00,0.00}{addition}\\
\[(P,Q) \mapsto P \textcolor[rgb]{1.00,0.00,0.00}{+} Q\]
An element $P$ generates a cyclic Subgroup with order $n$\\
\[\langle P \rangle = \{\textcolor[rgb]{1.00,0.00,0.00}{k}P : k \in \mathbb{Z} \} \]
\[n = ord(P) = |\langle P \rangle| \]\\ \vspace{0.5cm}
\textbf{The Discrete Logarithm Problem (DLP)}: \\
For a given $P$ and $Q \in \langle P \rangle$ determine $k$ s.t. \\
\[Q=\textcolor[rgb]{1.00,0.00,0.00}{k}P\]

\end{frame}
\begin{frame}{Make the DLP hard}
\textbf{The Discrete Logarithm Problem (DLP)}: \\
For a given $P$ and $Q \in \langle P \rangle$ determine $k$ s.t. \\
\[Q=P^k\]
\
\textbf{Brute-force attack:}
try every $k < n$
\begin{itemize}
	\item big group
	\item big $n=ord(\langle P \rangle)$ 
\end{itemize}

\pause
\vspace{1cm}
$(\mathbb{F}_p, \cdot)$ \pause is used in:
\begin{itemize}
\item Digital Signature Algorithm (DSA)
\item Diffie-Hellman (DH)
\item El-Gamal
\item \textcolor{gray}{RSA (on IFC)}
\end{itemize}
\end{frame}
\begin{frame}
	\frametitle{RSA \& DSA Keysize}
	\begin{center}
		\begin{tabular}{|c|c|c|c|}
			\hline
			\thead{Security \\strength}& \thead{Symmetric \\algorithm} & \thead{FFC(DSA)}& \thead{IFC(RSA)} \\ \hline
			\cellcolor{red!25}$\leq 80$ & 2TDEA & \makecell{$L=1024$\\$N=160$} & \cellcolor{green!25}$K=1024$ \\ \hline  
			\cellcolor{orange!25}$112$ & 3TDEA & \makecell{$L=2048$\\$N=224$} & \cellcolor{green!25}$K=2048$ \\ \hline  
			\cellcolor{orange!25}$128$ & AES-128 & \makecell{$L=3072$\\$N=256$} & \cellcolor{green!25}$K=3072$ \\ \hline  
			\cellcolor{green!25}$192$ & AES-192 & \makecell{$L=7680$\\$N=384$} & \cellcolor{red!25}$K=7680$ \\ \hline  
			\cellcolor{green!25}$256$ & AES-256 & \makecell{$L=15360$\\$N=512$} & \cellcolor{red!25}$K=15360$ \\ \hline  
		\end{tabular}
		\captionof{table}{Security Strength of DSA and RSA from NIST\cite{barker2016recommendation}}
	\end{center}
	
\end{frame}
%\begin{frame}
%	\frametitle{Handshake size with RSA}
%	\begin{center}
%		\begin{tabular}{|c|c|c|c|}
%			\hline
%			\thead{RSA \\key \\(bits)}& \thead{X.509 \\certificate \\(bytes)} & \thead{Handshake,\\ no chain \\(bytes)}& \thead{Handshake, \\chain \\(bytes)} \\ \hline
%			$1024$ & \makecell{$ 589 $} & $ 1225 $ & $ 2073 $ \\ \hline  
%			$2048$ & \makecell{$ 845 $} & $ 1481 $ & $ 2585 $ \\ \hline  
%			$3072$ & \makecell{$ 1101 $} & $ 1737 $ & $ 3097 $ \\ \hline  
%			$4096$ & \makecell{$ 1357 $} & $ 1993 $ &  \\ \hline
%		\end{tabular} 
%		\captionof{table}{Sizes of handshakes and certificates in with RSA\cite{cuppens2015fps}}
%	\end{center}
%	
%\end{frame}
\begin{frame}
\frametitle{Continuous Eliptic Curve}
\begin{figure}
	\centering
	\def\svgwidth{\columnwidth}
	\input{img/ECClines-3.pdf_tex}
	\caption{Two eliptic curves over $\mathbb{R}$ \cite{wikiYassineMrabet}}
\end{figure}
\begin{itemize}
	\pause
	\item Symmetry axis: x-axis
	\pause
	\item Every line intersecting two points has a third intersection point
	\pause
	\item Vertical lines intersect "infinity"
\end{itemize}
\end{frame}
\begin{frame}
\frametitle{Continuous Eliptic Curve}
\begin{figure}
	\centering
	\def\svgwidth{\columnwidth}
	\input{img/ECClines-3.pdf_tex}
	\caption{Two eliptic curves over $\mathbb{R}$ \cite{wikiYassineMrabet}}
\end{figure}
\begin{itemize}
	\pause
	\item Element: point on the curve
	\pause
	\item $+$: 3rd intersection of a line, reflect over the x-axis
	\pause
	\item Neutral element: "infinity", $O$
\end{itemize}
\end{frame}
\begin{frame}
	\frametitle{Continuous Eliptic Curve}
	\begin{figure}
		\centering
		\def\svgwidth{\columnwidth}
		\input{img/ECClines-3.pdf_tex}
		\caption{Two eliptic curves over $\mathbb{R}$ \cite{wikiYassineMrabet}}
	\end{figure}
	\begin{itemize}
		\item Rounding errors
		\item Not suitable for cryptography
	\end{itemize}
\end{frame}


\begin{frame}
	\frametitle{Eliptic curves over finite fields}
	\begin{itemize}
		\item Make it discrete!
		\item "Random" jumps through a set of points
	\end{itemize}
	\begin{figure}
		\centering
		\def\svgwidth{\columnwidth}
		\input{img/Elliptic_curve_on_Z61.pdf_tex}
		\caption{Set of affine points of elliptic curve $y^2 = x^3 - x$ over finite field $\mathbb{F}_{61}$.}
	\end{figure}
\end{frame}
\begin{frame}
	\frametitle{ECC key sizes}
	\begin{center}
		\begin{tabular}{|c|c|c|}
			\hline
			\thead{Security \\strength}& \thead{IFC(RSA)} & \thead{ECC} \\ \hline
			\makecell{$\leq 80$} & $k=1024$ & $f=160-223$ \\ \hline  
			\makecell{$112$} & $k=2048$ & $f=224-255$ \\ \hline  
			\makecell{$128$} & $k=3072$ & $f=256-383$ \\ \hline  
			\makecell{$192$} & $k=7680$ & \cellcolor{green!25}$f=384-511$ \\ \hline  
			\makecell{$256$} & $k=15360$ & \cellcolor{green!25}$f=512+$ \\ \hline  
		\end{tabular}
		\captionof{table}{Security Strength of ECC compared to RSA\cite{barker2016recommendation}}
	\end{center}
\end{frame}
\begin{frame}
	\frametitle{Handshake size of ECC compared to RSA}
	\begin{center}
		\begin{tabular}{|c|c|c|c|}
			\hline
			\thead{RSA \\key \\(bits)}& \thead{X.509 \\certificate \\(bytes)} & \thead{Handshake,\\ no chain \\(bytes)}& \thead{Handshake, \\chain \\(bytes)} \\ \hline
			$1024$ & $ 589 $ & $ 1225 $ & $ 2073 $ \\ \hline  
			$2048$ & $ 845 $ & $ 1481 $ & $ 2585 $ \\ \hline  
			$3072$ & $ 1101 $ & $ 1737 $ & $ 3097 $ \\ \hline  
			$4096$ & $ 1357 $ & $ 1993 $ &  \\ \hline
		\end{tabular} 
		\begin{tabular}{|c|c|c|c|}
			\hline
			\thead{ECC \\key \\(bits)}& \thead{X.509 \\certificate \\(bytes)} & \thead{Handshake,\\ no chain \\(bytes)}& \thead{Handshake, \\chain \\(bytes)} \\ \hline
			$160$ & $ 291 $ & $ 959 $ & $ 1277 $ \\ \hline  
			$224$ & $ 315 $ & $ 983 $ & $ 1317 $ \\ \hline  
			$256$ & $ 331 $ & $ 999 $ & $ 1349 $ \\ \hline  
			$288$ & $ 347 $ & $ 1015 $ &  \\ \hline
		\end{tabular} 
		\captionof{table}{Sizes of handshakes and certificates with ECC and RSA\cite{cuppens2015fps}}
	\end{center}
\end{frame}
\begin{frame}
	\frametitle{NIST-Curves}
	The National Institute for Standards and Technology (NIST) proposed some cryptographic curves in 1999.
	\begin{itemize}
		\item Special characteristics for efficiency
		\item Chosen ''randomly''
	\end{itemize}
	\vspace{1cm}
	\textbf{ECC is not as hard as ECDLP!} \\
	Some attacks can be performed on special classes of curves.
	\begin{itemize}
		\item Surprise! Attacks on NIST-Curves have been found
		\item NIST-Curves were probably not chosen at random
	\end{itemize}
	\cite{bernstein2016safecurves}
\end{frame}
\begin{frame}
	\frametitle{Alternatives: Curve25519, M-511, M-383, ...}
	Curve25519:
	\begin{itemize}
		\item Proposed by Daniel Bernstein \cite{bernstein2006curve25519}
		\item No security flaws found until today
		\item De facto standard implemented in most libraries
	\end{itemize}
	\vspace{1cm}
	M-511, M-383, M-221, E-521, E-382, E-222:
	\begin{itemize}
		\item Proposed by Diego F. Aranha et. al. \cite{cryptoeprint:2013:647}
		\item No security flaws found until today
	\end{itemize}
\end{frame}
	\begin{frame}
		\frametitle{Term Project}
		\textbf{Goals for the semester:}
		\begin{itemize}
			\item understand the maths behind ECC
			\item implement a library with M-511 with:
			\begin{itemize}
				\item key generation
				\item en-/decryption
				\item signature and verification
			\end{itemize}
		\end{itemize}
	
		\vspace{0.8cm}
		''Never implement your own crypto'' \\
		\vspace{0.8cm}
		\textbf{We will not:}
		\begin{itemize}
			\item Implement a library for real-world use
			\item Care about side-channel attacks
		\end{itemize}
	\end{frame}

%\section{Abschnitt Nr. 1} 
%\begin{frame}
%\frametitle{Titel} 
%Die einzelnen Frames sollte einen Titel haben 
%\end{frame}
%\subsection{Unterabschnitt Nr.1.1  }
%\begin{frame} 
%Denn ohne Titel fehlt ihnen was
%\end{frame}
%
%
%\section{Abschnitt Nr. 2} 
%\subsection{Listen I}
%\begin{frame}\frametitle{Aufz\"ahlung}
%\begin{itemize}
%\item Einf\"uhrungskurs in \LaTeX  
%\item Kurs 2  
%\item Seminararbeiten und Pr\"asentationen mit \LaTeX 
%\item Die Beamerclass 
%\end{itemize} 
%\end{frame}
%
%\begin{frame}\frametitle{Aufz\"ahlung mit Pausen}
%\begin{itemize}
%\item  Einf\"uhrungskurs in \LaTeX \pause 
%\item  Kurs 2 \pause 
%\item  Seminararbeiten und Pr\"asentationen mit \LaTeX \pause 
%\item  Die Beamerclass
%\end{itemize} 
%\end{frame}
%
%\subsection{Listen II}
%\begin{frame}\frametitle{Numerierte Liste}
%\begin{enumerate}
%\item  Einf\"uhrungskurs in \LaTeX 
%\item  Kurs 2
%\item  Seminararbeiten und Pr\"asentationen mit \LaTeX 
%\item  Die Beamerclass
%\end{enumerate}
%\end{frame}
%\begin{frame}\frametitle{Numerierte Liste mit Pausen}
%\begin{enumerate}
%\item  Einf\"uhrungskurs in \LaTeX \pause 
%\item  Kurs 2 \pause 
%\item  Seminararbeiten und Pr\"asentationen mit \LaTeX \pause 
%\item  Die Beamerclass
%\end{enumerate}
%\end{frame}
%
%\section{Abschnitt Nr.3} 
%\subsection{Tabellen}
%\begin{frame}\frametitle{Tabellen}
%\begin{tabular}{|c|c|c|}
%\hline
%\textbf{Zeitpunkt} & \textbf{Kursleiter} & \textbf{Titel} \\
%\hline
%WS 04/05 & Sascha Frank &  Erste Schritte mit \LaTeX  \\
%\hline
%SS 05 & Sascha Frank & \LaTeX \ Kursreihe \\
%\hline
%\end{tabular}
%\end{frame}
%
%
%\begin{frame}\frametitle{Tabellen mit Pause}
%\begin{tabular}{c c c}
%A & B & C \\ 
%\pause 
%1 & 2 & 3 \\  
%\pause 
%A & B & C \\ 
%\end{tabular} 
%\end{frame}
%
%
%\section{Abschnitt Nr. 4}
%\subsection{Bl\"ocke}
%\begin{frame}\frametitle{Bl\"ocke}
%
%\begin{block}{Blocktitel}
%Blocktext 
%\end{block}
%
%\begin{exampleblock}{Blocktitel}
%Blocktext 
%\end{exampleblock}
%
%
%\begin{alertblock}{Blocktitel}
%Blocktext 
%\end{alertblock}
%\end{frame}
%
%\section{Abschnitt Nr. 5}
%\subsection{Geteilter Bildschirm}
%
%\begin{frame}\frametitle{Zerteilen des Bildschirmes}
%\begin{columns}
%\begin{column}{4cm}
%\begin{itemize}
%\item Beamer 
%\item Beamer Class 
%\item Beamer Class Latex 
%\end{itemize}
%\end{column}
%\begin{column}{5cm}
%\begin{tabular}{|c|c|}
%\hline
%\textbf{Kursleiter} & \textbf{Titel} \\
%\hline
%Sascha Frank &  Kurs 1 \\
%\hline
%Sascha Frank &  Kursreihe \\
%\hline
%\end{tabular}
%\end{column}
%\end{columns}
%\end{frame}
%
%\subsection{Bilder} 
%\begin{frame}\frametitle{Bilder in Beamer}
%\begin{figure}
%%\includegraphics[scale=0.45]{PIC1} 
%\caption{Die Abbildung zeigt ein Beispielbild}
%\end{figure}
%\end{frame}
%
%
%\subsection{Bilder und Listen kombiniert} 
%
%\begin{frame}
%\frametitle{Bilder und Itemization in Beamer}
%\begin{columns}
%\begin{column}{4cm}
%\begin{itemize}
%\item<1-> Stichwort 1
%\item<3-> Stichwort 2
%\item<5-> Stichwort 3
%\end{itemize}
%\vspace{3cm} 
%\end{column}
%\begin{column}{5cm}
%\begin{overprint}
%%\includegraphics<2>[scale=0.5]{PIC1}
%%\includegraphics<4>[scale=0.5]{PIC2}
%%\includegraphics<6>[scale=0.5]{PIC3}
%\end{overprint}
%\end{column}
%\end{columns}
%\end{frame}
%
%\subsection{Bilder die mehr Platz brauchen} 
%\begin{frame}[plain]
%\frametitle{plain, oder wie man mehr Platz hat}
%\begin{figure}
%%\includegraphics[scale=0.5]{PIC1} 
%\caption{Die Abbildung zeigt ein Beispielbild}
%\end{figure}
%\end{frame}




\begin{frame}[t,allowframebreaks]
	\frametitle{References}
	\printbibliography
\end{frame}


\end{document}